%----------------------------------------------------------------------------------------
%	SOLUTION 4
%----------------------------------------------------------------------------------------
\subsection*{Problem 4}
I propose to optimally combine the present radar system with a
new radar system that has a measurement variance of $6$. The reason is explained in the next paragraphs.


To address this problem, either we can use iterative least square or batch weighted least square and both leads to the same solution. I am using batch weighted least square where I assume that all the data from each sensor are available and processed simultaneously. Let the measurements $y \in \mathbb{R}^m$ coming from different sensors and true position $x \in \mathbb{R}^n$ are related as follows:
\begin{align*}
	y = Hx + e,
\end{align*}
where, $H \in \mathbb{R}^{m \times n}$ is the measurement matrix and $e \in \mathbb{R}^m$ is error matrix associated with $m$ measurements. Let the error covariance matrix is,
\begin{align*}
	R = \mathbb{E}[ee^T] = \begin{bmatrix}\sigma_1^2 & 0 & \ldots & 0\\ 0 & \sigma_2^2 & \ldots & 0\\ \vdots & \vdots & \vdots & \vdots\\0 & \ldots & 0 & \sigma_m^2 \end{bmatrix}.
\end{align*}
We know that weighted least square estimate of $x$ is:
\begin{align*}
	\hat{x} = (H^TR^{-1}H)^{-1}H^TR^{-1}y.
\end{align*}
The error associated with estimation of true position is
\begin{align*}
	\epsilon_x &= (x-\hat{x})\\
	&= x-(H^TR^{-1}H)^{-1}H^TR^{-1}y\\
	&= x-(H^TR^{-1}H)^{-1}H^TR^{-1}(Hx+e)\\
	&= x-x-(H^TR^{-1}H)^{-1}H^TR^{-1}e\\
	&= -(H^TR^{-1}H)^{-1}H^TR^{-1}e.
\end{align*}
Therefore, we can find the variance of estimation error as follows:
\begin{align}\label{eq:q4_exp_error}
	\mathbb{E}[\epsilon_x \epsilon_x^T] &= (H^TR^{-1}H)^{-1}H^TR^{-1}\mathbb{E}[ee^T]R^{-1}H(H^TR^{-1}H)^{-1}\\
	&= (H^TR^{-1}H)^{-1}H^TR^{-1}H(H^TR^{-1}H)^{-1}\\
	&= (H^TR^{-1}H)^{-1}.
\end{align}
\paragraph{Case 1:} When we have only one sensor with measure variance of $\sigma_1^2 = 10$, we have the following model
\begin{align*}
	y = \underbrace{[1]}_Hx + e
\end{align*}
and $R = [10]$. From~(\ref{eq:q4_exp_error}),
\begin{align*}
	\mathbb{E}[\epsilon_x \epsilon_x^T] = 10.
\end{align*}
\paragraph{Case 2:} When we combine a sensor having measurement variance of $\sigma_2^2 = 6$, with the current sensor with $\sigma_1^2 = 10$, we have the following model
\begin{align*}
y = \underbrace{\begin{bmatrix}1\\1\end{bmatrix}}_Hx + e
\end{align*}
and $R = \begin{bmatrix}10 & 0\\0 & 6\end{bmatrix}$. From~(\ref{eq:q4_exp_error}),
\begin{align*}
\mathbb{E}[\epsilon_x \epsilon_x^T] &= \left(\begin{bmatrix}1 & 1\end{bmatrix}\begin{bmatrix}\frac{1}{10} & 0\\0 & \frac{1}{6}\end{bmatrix}\begin{bmatrix}1\\1\end{bmatrix}\right)^{-1}\\
&= \left(\frac{1}{10} + \frac{1}{6}\right)^{-1}\\
&= 3.75.
\end{align*}
\paragraph{Case 3:} When we combine two sensors having same measurement variance of $10$, as the original system, along with the original one, we have the following model
\begin{align*}
y = \underbrace{\begin{bmatrix}1\\1\\1\end{bmatrix}}_Hx + e
\end{align*}
and $R = \begin{bmatrix}10 & 0 & 0\\0 & 10 & 0\\0 & 0 & 10\end{bmatrix} = 10 I_3$. From~(\ref{eq:q4_exp_error}),
\begin{align*}
	\mathbb{E}[\epsilon_x \epsilon_x^T] &= \left(\begin{bmatrix}1 & 1 & 1\end{bmatrix}\frac{1}{10}I_3\begin{bmatrix}1\\1\\1\end{bmatrix}\right)^{-1}\\
	&= \frac{10}{3}\\
	&= 3.33.
\end{align*}
Therefore, we can see that case 2 gives the least variance in error of position estimation. Hence, a good choice of sensors would be the combination of current sensor having measurement variance $10$ with another with measurement variance of $6$. 