%----------------------------------------------------------------------------------------
%	SOLUTION 2.15
%----------------------------------------------------------------------------------------
\subsection*{Solution 2.15}
\paragraph{2.15.a:} The pmf of Poisson random variable, $X$, with parameter $\lambda$, is given by
\begin{align*}
	P(X=k) = \frac{\lambda^ke^{-\lambda}}{k!},\ k=0,1,\ldots.
\end{align*}
The probability that at least one of the galaxies contains two or more black holes is given by
\begin{align*}
	P\left(\bigcup_{i=1}^n\{X_i \geq 2\}\right) &= 1- P\left(\bigcap_{i=1}^n\{X_i \leq 1\}\right)\\
	&= 1- \prod_{i=1}^{n}P(X_i \leq 1)\hspace*{1cm}[\text{as $X_i$ and $X_j$ are independent for all $i,j=1,2,\ldots,n, i \neq j$}]\\
	&= 1- \prod_{i=1}^n [P(X_i=0)+P(X_i=1)]\\
	&= 1- \prod_{i=1}^n [e^{-\lambda}+\lambda e^{-\lambda}]\\
	&= 1- [e^{-\lambda}+\lambda e^{-\lambda}]^n\\
	&= 1- e^{-n\lambda}[1+\lambda]^n.
\end{align*}
\paragraph{2.15.b:}The probability that all $n$ galaxies have at least one black hole is given by
\begin{align*}
	P\left(\bigcap_{i=1}^nP(X_i \geq 1)\right) &= \prod_{i=1}^n P(X_i \geq 1)\hspace*{1cm}[\text{as $X_i$ and $X_j$ are independent for all $i,j=1,2,\ldots,n, i \neq j$}]\\
	&= \prod_{i=1}^n[1-P(X_i =0)]\\
	&= \prod_{i=1}^n [1-e^{-\lambda}]\\
	&= [1-e^{-\lambda}]^n.
\end{align*}
\paragraph{2.15.c:}The probability that all $n$ galaxies have exactly on eblack hole is given by
\begin{align*}
	P\left(\bigcap_{i=1}^n P(X_i = 1)\right) &= \prod_{i=1}^n P(X_i=1)\\
	&= [\lambda e^{-\lambda}]^n\\
	&= \lambda^{n}e^{-n\lambda}.
\end{align*}
%----------------------------------------------------------------------------------------
%	SOLUTION 2.21
%----------------------------------------------------------------------------------------
\subsection*{Solution 2.21}
\paragraph{2.21.a:}
\begin{align*}
	P(X > n) &= \sum_{i=n+1}^{\infty}P(X_i=i)\\
	&= \sum_{i=n+1}^{\infty}(1-p)p^{i-1}\\
	&= \sum_{i=n+1}^{\infty}(p^{i-1}-p^i)\\
	&= \sum_{k=0}^{\infty}(p^{k+n}-p^{k+n+1})\\
	&= (1-p)p^n\sum_{k=0}^{\infty}p^k\\
	&= (1-p)p^n\frac{1}{1-p}\hspace*{1cm}[p \neq 1]\\
	&= p^n.
\end{align*}
This completes the proof. $\square$
\paragraph{2.21.b:}
\begin{align*}
	P(X > n+k|X > n) &= \frac{P(\{X > n+k\}\cap \{X > n\})}{P(X > n)}\hspace*{1cm}[\text{from Bayes rule}]\\
	&= \frac{P(X > n+k)}{P(X > n)}\hspace*{1cm}[\text{as $\{X>n+k\}\subseteq\{X>n\}$}]\\
	&= \frac{\sum\limits_{i=n+k+1}^{\infty}P(X=i)}{\sum\limits_{i=n+1}^{\infty}P(X=i)}\\
	&= \frac{\sum\limits_{i=n+k+1}^{\infty}(1-p)p^{i-1}}{\sum\limits_{i=n+1}^{\infty}(1-p)p^{i-1}}\\
	&= \frac{p^{n+k}}{p^{n}}\hspace*{1cm}[\text{from $2.21.a$}]\\
	&= p^k.
\end{align*}